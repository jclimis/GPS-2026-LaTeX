\documentclass[11pt, a4paper]{article}
\usepackage{formato_catedra}
% ─── Paquetes adicionales ────────────────────────────────────────────────────


% ─── Configuración de colores temáticos ─────────────────────────────────────
\definecolor{azulCatedra}{RGB}{26,82,118}
\definecolor{verdeCatedra}{RGB}{20,90,50}
\definecolor{amarilloAviso}{RGB}{251,243,219}
\definecolor{rojoError}{RGB}{231,76,60}
\definecolor{azulClaro}{RGB}{213,234,248}
\definecolor{verdeClaro}{RGB}{210,240,220}
\definecolor{grisCabecera}{RGB}{52,73,94}
\definecolor{grisFilaImpar}{RGB}{245,247,250}

% ─── Estilos de cajas especiales ────────────────────────────────────────────
\tcbuselibrary{skins,breakable}

\newtcolorbox{cajaResolucion}[1][]{
    enhanced, breakable,
    colback=verdeClaro!60,
    colframe=verdeCatedra,
    fonttitle=\bfseries\small,
    title={$\checkmark$ Resolución},
    #1
}

\newtcolorbox{cajaError}[1][]{
    enhanced, breakable,
    colback=amarilloAviso,
    colframe=rojoError,
    coltitle=white,
    fonttitle=\bfseries\small,
    title={\textbf{[!] Errores Frecuentes}},
    #1
}

\newtcolorbox{cajaFormula}[1][]{
	enhanced,
	colback=azulClaro!50,
	colframe=azulCatedra,
	fonttitle=\bfseries\small,
	title={Fórmula},
	halign=center,       % <--- Centrado automático de todo el contenido
	sharp corners=south, % Opcional: un toque estético
	before skip=10pt,
	after skip=10pt,
	#1
}


\newtcolorbox{cajaEnunciado}[1][]{
    enhanced, breakable,
    colback=white,
    colframe=azulCatedra,
    fonttitle=\bfseries,
    #1
}

% ─── Comandos auxiliares de tabla ────────────────────────────────────────────
\newcommand{\cabTbl}[1]{\cellcolor{grisCabecera}\textcolor{white}{\textbf{#1}}}
\newcommand{\subCabTbl}[1]{\cellcolor{azulCatedra!80}\textcolor{white}{\textbf{\small #1}}}

% ─── CONFIGURACIÓN DEL MODO (Enunciado / Resolución) ─────────────────────────
% Cambia a \resolucionfalse para generar sólo el enunciado
\newif\ifresolucion
\resolucionfalse

% ─── Metadatos del documento ─────────────────────────────────────────────────
\Lectura{Práctico II\_P1}

\ifresolucion
    \TituloDocumento{ESTIMACIÓN POR PUNTOS DE CASOS DE USO\\[4pt]
                     {\normalsize (con Resolución)}}
\else
    \TituloDocumento{ESTIMACIÓN POR PUNTOS DE CASOS DE USO\\[4pt]
                     {\normalsize (Enunciado)}}
\fi

% ─────────────────────────────────────────────────────────────────────────────
\begin{document}
% ─────────────────────────────────────────────────────────────────────────────

\HacerTitulo

% ─────────────────────────────────────────────────────────────────────────────
\section*{1 \quad CONTENIDO}
% ─────────────────────────────────────────────────────────────────────────────
Módulo II. Planificación y Gestión de Proyectos de Software. Actividades prácticas del
tema \textbf{Estimación mediante Puntos de Casos de Uso (PCU)}.

% ─────────────────────────────────────────────────────────────────────────────
\section*{2 \quad OBJETIVOS}
% ─────────────────────────────────────────────────────────────────────────────
\begin{itemize}[leftmargin=1.5cm, label=--]
    \item \textbf{Aplicar} una técnica de estimación basada en el paradigma
          orientado a objetos.
    \item \textbf{Comprender} los conceptos de complejidad de actores, complejidad
          de casos de uso, factores técnicos y factores de entorno.
    \item \textbf{Calcular} los Puntos de Casos de Uso sin ajuste (UUCP), el Factor
          de Complejidad Técnica (TCF), el Factor de Entorno (ECF) y los PCU ajustados
          (AUCP).
    \item \textbf{Estimar} el esfuerzo de desarrollo mediante los modelos de Karner,
          Schneider \& Winters y productividades propias.
    \item \textbf{Afianzar} los conceptos de caso de uso al nivel de \emph{meta de
          usuario}, escenario principal exitoso y extensiones.
\end{itemize}

% ─────────────────────────────────────────────────────────────────────────────
\section*{3 \quad METODOLOGÍA}
% ─────────────────────────────────────────────────────────────────────────────
Resolución de ejercicios prácticos mediante la aplicación de conceptos expuestos en
las clases teóricas, con base en la bibliografía propuesta. Se presentan dos sistemas
reales simplificados (ATM y TPV) para ejercitar la técnica de estimación
PCU en forma completa, desde la clasificación de actores hasta la duración del proyecto.

% ─────────────────────────────────────────────────────────────────────────────
\section*{4 \quad BIBLIOGRAFÍA}
% ─────────────────────────────────────────────────────────────────────────────
\nocite{karner1993,cohn2013ucp,anda2001industry,pressman2021_cap25,
        schneider1998applying,cockburn2001writing}
\bibliographystyle{plain}
\bibliography{referencias}

\vspace{0.5em}
\noindent\textbf{Nota bibliográfica:} La productividad de referencia de Karner
(20\,h/PCU) se publicó originalmente en~\cite{karner1993}. Análisis posteriores
\cite{anda2001industry} y la experiencia de industria documentada en
\cite{cohn2013ucp} muestran que este valor puede variar ampliamente; se recomienda
calibrarlo con datos históricos propios de la organización.

% ─────────────────────────────────────────────────────────────────────────────
\section*{5 \quad ACTIVIDADES}
% ─────────────────────────────────────────────────────────────────────────────

% ═══════════════════════════════════════════════════════════════════════════════
\subsection*{5.1 \quad Problema 1: Análisis de la Bibliografía}
% ═══════════════════════════════════════════════════════════════════════════════

Leer los artículos propuestos~\cite{cohn2013ucp,anda2001industry} y responder
a las siguientes preguntas:

\begin{enumerate}[label=\textbf{A1.\arabic*.}, leftmargin=1.8cm, itemsep=0.8em]

\item \textbf{¿Qué miden los Puntos de Caso de Uso (PCU) y de qué son
      función?}~\cite{cohn2013ucp}

\ifresolucion
\begin{cajaResolucion}
Los PCU miden el \textbf{tamaño funcional} del software desde la perspectiva del
usuario, expresado a través de los casos de uso. Son función de la
\textbf{complejidad relativa} de los actores y de los casos de uso del sistema
(cantidad de transacciones por CU), ajustada por factores técnicos del proyecto
y del entorno del equipo de desarrollo.
\end{cajaResolucion}
\fi

\item \textbf{El autor~\cite{cohn2013ucp} asimila la \textsc{transacción} a otro
      concepto cuando el caso de uso se especifica en un nivel adecuado. ¿Cuál es
      ese concepto y cuál es el nivel de descripción adecuado para que la
      equivalencia sea válida?}

\ifresolucion
\begin{cajaResolucion}
El autor asimila la \emph{transacción} al concepto de \textbf{paso} (o
\emph{action step}) dentro de la descripción del caso de uso. La equivalencia
es válida únicamente cuando el caso de uso se especifica al nivel de
\textbf{meta de usuario} (\emph{user goal}): cada paso del escenario principal
representa una interacción completa entre actor y sistema.

\begin{cajaError}
\textbf{Error frecuente:} confundir ``transacción'' con ``caso de uso
completo''. La transacción es \emph{un paso} del CU, no el CU en sí.
Contar el CU entero como una sola transacción produce estimaciones muy
subajustadas.
\end{cajaError}
\end{cajaResolucion}
\fi

\item \textbf{¿Qué es la \textsc{meta} de un caso de uso, un
      \textsc{escenario}, un \textsc{escenario principal exitoso} y una
      \textsc{extensión}, según~\cite{cohn2013ucp}?}

\ifresolucion
\begin{cajaResolucion}
\begin{itemize}[leftmargin=1cm, itemsep=0.3em]
  \item \textbf{Meta:} el objetivo concreto que el actor principal intenta
        alcanzar al interactuar con el sistema. Define el \emph{nivel} del CU.
  \item \textbf{Escenario:} una secuencia específica de acciones e interacciones
        entre actores y el sistema; es una instancia de un CU.
  \item \textbf{Escenario principal exitoso (EPE):} la secuencia ``feliz'' de
        pasos que lleva directamente a la meta, sin errores ni alternativas.
  \item \textbf{Extensión:} condición alternativa o excepcional que bifurca el
        flujo normal; puede o no conducir al éxito.
\end{itemize}
\end{cajaResolucion}
\fi

\item \textbf{¿Qué es una \textsc{transacción} según~\cite{anda2001industry}?}

\ifresolucion
\begin{cajaResolucion}
Para Anda~\cite{anda2001industry} una transacción es \textbf{un conjunto de
acciones ejecutadas de forma atómica}: inicia con una entrada del actor, el
sistema procesa e interactúa con datos u otros sistemas, y concluye con una
respuesta al actor. Es la unidad de conteo básica para clasificar la
complejidad de un caso de uso.
\end{cajaResolucion}
\fi

\item \textbf{¿Qué problemas identifica~\cite{anda2001industry} para contar el
      número de transacciones en un caso de uso?}

\ifresolucion
\begin{cajaResolucion}
Anda identifica los siguientes problemas principales:
\begin{itemize}[leftmargin=1cm, itemsep=0.3em]
  \item \textbf{Granularidad inconsistente} de la especificación: distintos
        analistas escriben los CU con niveles de detalle muy diferentes.
  \item \textbf{Ambigüedad en las extensiones}: no queda claro si los pasos
        alternativos deben contarse como transacciones adicionales.
  \item \textbf{Relaciones entre CU} (\emph{include}/\emph{extend}): dificultan
        delimitar qué pasos pertenecen a qué CU.
  \item \textbf{Falta de experiencia del estimador} para mapear pasos de CU a
        la definición de transacción.
\end{itemize}
\end{cajaResolucion}
\fi

\item \textbf{Según~\cite{cohn2013ucp}, ¿las extensiones deben incluirse en el
      cálculo de PCU o no? ¿Por qué?}

\ifresolucion
\begin{cajaResolucion}
\textbf{No deben incluirse} las extensiones en el conteo de transacciones para
clasificar la complejidad del CU (Cohn, p.~8). El razonamiento es que las
extensiones representan flujos alternativos o de excepción que, aunque añaden
esfuerzo de prueba, en general no cambian sustancialmente el
\emph{peso} del CU en la estimación. Incluirlas inflaría artificialmente la
complejidad y haría que CU simples pasasen a categorías superiores, distorsionando
la estimación total.
\end{cajaResolucion}
\fi

\item \textbf{Para el cálculo de la duración se propuso originalmente
      (Karner~\cite{karner1993}) 20\,h por cada PCU. ¿Puede mantenerse
      actualmente esa correspondencia? ¿Qué puede hacerse al respecto
      (\cite{cohn2013ucp})?}

\ifresolucion
\begin{cajaResolucion}
\textbf{No puede mantenerse directamente.} Karner definió el valor de 20\,h/PCU
en 1993, basándose en datos de proyectos de la época con herramientas,
lenguajes y metodologías muy distintos a los actuales. En 30 años las
productividades han cambiado radicalmente:

\begin{center}
\small
\begin{tabular}{lcc}
\toprule
\textbf{Productividad} & \textbf{h-h / PCU} & \textbf{Referencia} \\
\midrule
Baja (equipos inexpertos) & 28--36 & Schneider \& Winters \\
Media (Karner original)   & 20     & Karner (1993) \\
Alta (equipos actuales)   & 4--10  & Estadísticas recientes \\
\bottomrule
\end{tabular}
\end{center}

\textbf{¿Qué hacer?} Cohn recomienda \textbf{calibrar la productividad} con
datos históricos de proyectos anteriores similares. La fórmula correcta es:
\[
  \text{Productividad} = \frac{\text{horas reales de codificación}}
                               {\text{AUCP del proyecto}}
\]
Una vez disponible ese valor propio, el modelo PCU deja de ser especulativo y
se convierte en una herramienta predictiva confiable.

\begin{cajaError}
\textbf{Error frecuente A1.7:} Aplicar la conversión $44.000\,\text{h}/2.000\,\text{h}
= 22$\,h/PCU en lugar de $44.000/2.000 = 22$ (división correcta). El valor
de referencia Sch\&Win ($800\,\text{h}$) se usa para otro contexto de
normalización.
\end{cajaError}
\end{cajaResolucion}
\fi

\end{enumerate}

% ═══════════════════════════════════════════════════════════════════════════════
\subsection*{5.2 \quad Problema 2: Sistema ATM Simplificado}
% ═══════════════════════════════════════════════════════════════════════════════

Obtener los Puntos de Casos de Uso (PCU sin ajuste y ajustados) para el sistema
descrito a continuación. Estimar el esfuerzo en h-h y meses-hombre usando
Karner, Schneider\,\&\,Winters y productividad propia.

% ─── 5.2.1 Definición ────────────────────────────────────────────────────────
\subsubsection*{5.2.1 \quad Definición del Problema}

Se requiere desarrollar un \textbf{prototipo elemental} que muestre las
funcionalidades más básicas de un cajero automático (ATM). El prototipo servirá
para descubrir requisitos, estudiar la implementación de funcionalidades y
ejercitar el patrón arquitectónico \textbf{Modelo--Vista--Control (MVC)}.

% ─── 5.2.2 Requisitos ────────────────────────────────────────────────────────
\subsubsection*{5.2.2 \quad Requisitos Funcionales y Restricciones}

\noindent\textbf{Funcionalidades:}
\begin{enumerate}[label=\arabic*., leftmargin=1.5cm]
    \item Validar PIN del cliente.
    \item Elegir Opciones de servicio.
    \item Consultar Saldo de una cuenta simulada.
    \item Extraer dinero.
    \item Depositar dinero.
    \item Transferir dinero entre dos cuentas.
\end{enumerate}

\noindent\textbf{Restricciones:}
\begin{itemize}[leftmargin=1.5cm, label=--]
    \item Funciona en una única PC, sin acceso a otros sistemas ni seguridad real.
    \item Sin procesamiento concurrente.
    \item Instalación rápida y mantenimiento muy sencillo (módulos reutilizables).
    \item Facilidad de uso esencial (intuitivo, sin formación previa).
    \item \textbf{Portabilidad esencial} $\Rightarrow$ lenguaje Java.
    \item Eficiencia y tiempo de respuesta \emph{no} son críticos.
\end{itemize}

% ─── 5.2.3 Características del Grupo ─────────────────────────────────────────
\subsubsection*{5.2.3 \quad Características del Grupo de Desarrollo}

\begin{cajaEnunciado}[title=Perfil del equipo – ATM]
Una única persona a tiempo parcial (\emph{part-time}), familiarizada con el
Proceso Unificado de Desarrollo (RUP), pero \textbf{sin experiencia en Java}.
\end{cajaEnunciado}

% ─── 5.2.4 Diagrama de Casos de Uso ──────────────────────────────────────────
\subsubsection*{5.2.4 \quad Diagrama de Casos de Uso del ATM}

\begin{figure}[H]
    \centering
    \includegraphics[width=0.88\textwidth]{images/diagrama_cu_atm.png}
    \caption{Diagrama de Casos de Uso del Sistema ATM Simplificado}
    \label{fig:cu_atm}
\end{figure}

% ─── 5.2.5 Especificación de Casos de Uso ────────────────────────────────────
\subsubsection*{5.2.5 \quad Especificación de Casos de Uso}

\noindent\textbf{CU\#01 -- Validar PIN}\\[2pt]
\begin{tabular}{@{}p{3cm}p{11cm}@{}}
\textit{Actor principal:} & Cliente \\
\textit{Precondiciones:}  & Ninguna. \\
\textit{Disparador:}      & El Cliente se presenta ante el ATM y activa el programa. \\
\end{tabular}

\smallskip\noindent\textit{Descripción (Escenario Principal Exitoso):}
\begin{enumerate}[label=\arabic*., leftmargin=1.5cm, noitemsep]
    \item Se despliega la pantalla de autenticación; el Cliente escribe usuario y PIN.
    \item El sistema valida el par (usuario, PIN) en la BDD y da acceso. Continúa en CU02.
\end{enumerate}
\textit{Extensiones:}
\begin{itemize}[leftmargin=1.5cm, noitemsep, label=--]
    \item \textbf{2a.} PIN inválido $\to$ se emite mensaje de invalidez; continúa en CU01, paso 1.
\end{itemize}

\medskip\noindent\textbf{CU\#02 -- Elegir Opciones}\\[2pt]
\begin{tabular}{@{}p{3cm}p{11cm}@{}}
\textit{Actor principal:} & Cliente \\
\textit{Precondiciones:}  & Par (usuario, PIN) válido. \\
\textit{Disparador:}      & El PIN acaba de ser validado (fin de CU01). \\
\end{tabular}

\smallskip\noindent\textit{Descripción:}
\begin{enumerate}[label=\arabic*., leftmargin=1.5cm, noitemsep]
    \item Se despliega la pantalla de opciones; el Cliente selecciona ``SALDO''. Continúa en CU03.
\end{enumerate}
\textit{Extensiones:}
\begin{itemize}[leftmargin=1.5cm, noitemsep, label=--]
    \item \textbf{1a.} ``EXTRAER'' $\to$ continúa en CU04. \quad
          \textbf{1b.} ``DEPOSITAR'' $\to$ continúa en CU05.
    \item \textbf{1c.} ``TRANSFERIR'' $\to$ continúa en CU06. \quad
          \textbf{1d.} ``SALIR'' $\to$ sistema se desactiva.
\end{itemize}

\medskip\noindent\textbf{CU\#03 -- Consultar Saldo}\\[2pt]
\textit{Precondiciones:} Par (usuario, PIN) válido. \textit{Disparador:} opción ``SALDO'' en CU02.
\begin{enumerate}[label=\arabic*., leftmargin=1.5cm, noitemsep]
    \item Pantalla de consulta. El Cliente introduce su número de cuenta.
    \item El sistema consulta la BDD y muestra el saldo.
    \item El Cliente selecciona ``SALIR''; se cierra la pantalla. Continúa en CU02.
\end{enumerate}
\textit{Extensiones:} \textbf{2a.} Cuenta inválida $\to$ mensaje de error; continúa en CU03 paso 1.

\medskip\noindent\textbf{CU\#04 -- Extraer}\\[2pt]
\textit{Precondiciones:} PIN válido. \textit{Disparador:} opción ``EXTRAER'' en CU02.
\begin{enumerate}[label=\arabic*., leftmargin=1.5cm, noitemsep]
    \item Pantalla de extracción. Cliente introduce cuenta y monto.
    \item Sistema verifica cuenta válida y saldo suficiente en BDD.
    \item Sistema actualiza saldo, muestra nuevo saldo, entrega dinero y comprobante.
    \item Cliente selecciona ``SALIR''. Continúa en CU02.
\end{enumerate}
\textit{Extensiones:} \textbf{2a.} Cuenta inválida $\to$ mensaje; regresa a paso 1. \quad
\textbf{2b.} Saldo insuficiente $\to$ ídem.

\medskip\noindent\textbf{CU\#05 -- Depositar}\\[2pt]
\textit{Precondiciones:} PIN válido. \textit{Disparador:} opción ``DEPOSITAR'' en CU02.
\begin{enumerate}[label=\arabic*., leftmargin=1.5cm, noitemsep]
    \item Pantalla de depósito. Cliente introduce cuenta y monto (más dinero físico).
    \item Sistema verifica cuenta y correspondencia entre monto y dinero físico.
    \item Sistema actualiza saldo, muestra nuevo saldo, entrega comprobante.
    \item Cliente selecciona ``SALIR''. Continúa en CU02.
\end{enumerate}
\textit{Extensiones:} \textbf{2a.} Cuenta inválida $\to$ devuelve dinero; regresa paso 1. \quad
\textbf{2b.} Monto no coincide $\to$ mensaje; regresa paso 1.

\medskip\noindent\textbf{CU\#06 -- Transferir}\\[2pt]
\textit{Precondiciones:} PIN válido. \textit{Disparador:} opción ``TRANSFERIR'' en CU02.
\begin{enumerate}[label=\arabic*., leftmargin=1.5cm, noitemsep]
    \item Pantalla de transferencia. Cliente introduce cuenta origen, destino y monto.
    \item Sistema verifica ambas cuentas y saldo suficiente en BDD.
    \item Sistema actualiza saldos en BDD, muestra nuevo saldo y emite comprobante.
    \item Cliente selecciona ``SALIR''. Continúa en CU02.
\end{enumerate}
\textit{Extensiones:} \textbf{2a.} Cuenta origen inválida. \quad
\textbf{2b.} Cuenta destino inválida. \quad
\textbf{2c.} Saldo insuficiente. En todos: mensaje y regresa paso 1.

% ─────────────────────────────────────────────────────────────────────────────
\ifresolucion
% ─────────────────────────────────────────────────────────────────────────────
\subsubsection*{Resolución -- Problema 2: Sistema ATM}

\begin{figure}[H]
    \centering
    \includegraphics[width=0.75\textwidth]{images/diagrama_pcu_metodologia.png}
    \caption{Metodología de cálculo PCU aplicada al Sistema ATM}
    \label{fig:metodologia_atm}
\end{figure}

\paragraph{Paso 1 y 2: Clasificación de Actores y UAW}

Los actores se clasifican según el tipo de interfaz con el sistema:

\begin{table}[H]
\centering
\small
\caption{Clasificación de Actores -- ATM}
\label{tab:actores_atm}
\begin{tabular}{p{3.5cm} p{4cm} c c c}
\toprule
\cabTbl{Actor} & \cabTbl{Tipo de Interfaz} & \cabTbl{Categoría} &
\cabTbl{Peso} & \cabTbl{Cantidad} \\
\midrule
\rowcolor{grisFilaImpar}
Cliente & Interfaz gráfica (GUI) & Complejo & 3 & 1 \\
BDD PIN & Archivo / base de datos & Simple & 1 & 1 \\
\rowcolor{grisFilaImpar}
BDD Cuenta & Archivo / base de datos & Simple & 1 & 1 \\
\midrule
\multicolumn{4}{r}{\textbf{UAW (Unadjusted Actor Weight)}} & \textbf{5} \\
\bottomrule
\end{tabular}
\end{table}

\begin{cajaFormula}
\[ \text{UAW} = 1\times 3 + 1\times 1 + 1\times 1 = \mathbf{5} \]
\end{cajaFormula}

\paragraph{Pasos 3 y 4: Clasificación de Casos de Uso y UUCW}

Se cuenta el número de transacciones del \emph{Escenario Principal Exitoso}
(sin extensiones) para clasificar cada CU.

\begin{table}[H]
\centering
\small
\caption{Clasificación de Casos de Uso -- ATM}
\label{tab:cu_atm}
\begin{tabular}{p{0.8cm} p{3cm} p{5cm} c c c}
\toprule
\cabTbl{CU\#} & \cabTbl{Nombre} & \cabTbl{Justificación} &
\cabTbl{Tx} & \cabTbl{Categoría} & \cabTbl{Peso} \\
\midrule
\rowcolor{grisFilaImpar}
01 & Validar PIN      & 2 pasos en EPE (autenticar + dar acceso) & 2 & Simple & 5 \\
02 & Elegir Opciones  & 1 paso en EPE (mostrar menú + elegir) & 1 & Simple & 5 \\
\rowcolor{grisFilaImpar}
03 & Consultar Saldo  & 2 pasos en EPE (ingresar cuenta + mostrar saldo) & 2 & Simple & 5 \\
04 & Extraer          & 2 pasos en EPE (verificar + actualizar/entregar) & 2 & Simple & 5 \\
\rowcolor{grisFilaImpar}
05 & Depositar        & 2 pasos en EPE (verificar + actualizar/recibo) & 2 & Simple & 5 \\
06 & Transferir       & 2 pasos en EPE (verificar ambas cuentas + actualizar) & 2 & Simple & 5 \\
\midrule
\multicolumn{5}{r}{\textbf{UUCW (Unadjusted Use Case Weight)}} & \textbf{30} \\
\bottomrule
\end{tabular}
\end{table}

\begin{itemize}[leftmargin=1.2cm, label=--]
    \item Simple: $\leq 3$ transacciones $\Rightarrow$ 5 puntos.
    \item Medio: 4--7 transacciones $\Rightarrow$ 10 puntos.
    \item Complejo: $>7$ transacciones $\Rightarrow$ 15 puntos.
\end{itemize}

\paragraph{Paso 5: Puntos de CU Sin Ajuste (UUCP)} \mbox{} \par

\begin{cajaFormula}
	$\displaystyle \text{UUCP} = \text{UAW} + \text{UUCW} = 5 + 30 = \mathbf{35}$
\end{cajaFormula}

\paragraph{Paso 6: Factor de Complejidad Técnica (TCF)}

La evaluación (0--5 ó 0--10) refleja el \textbf{grado de influencia} del factor
en el sistema ATM según las restricciones definidas.

\begin{table}[H]
\centering
\small
\caption{Factores de Complejidad Técnica -- ATM}
\label{tab:tcf_atm}
\begin{tabular}{c p{5.5cm} c c c}
\toprule
\cabTbl{Factor} & \cabTbl{Descripción} & \cabTbl{Peso ($w_i$)} &
\cabTbl{Evaluación ($e_i$)} & \cabTbl{$w_i\times e_i$} \\
\midrule
\rowcolor{grisFilaImpar}
T1 & Sistema distribuido             & 2.0 & 0 & 0.0 \\
T2 & Tiempo de respuesta (rendimiento) & 1.0 & 0 & 0.0 \\
\rowcolor{grisFilaImpar}
T3 & Eficiencia para el usuario final & 1.0 & 3 & 3.0 \\
T4 & Procesamiento interno complejo   & 1.0 & 0 & 0.0 \\
\rowcolor{grisFilaImpar}
T5 & Código reutilizable              & 1.0 & 5 & 5.0 \\
T6 & Facilidad de instalación         & 0.5 & 5 & 2.5 \\
\rowcolor{grisFilaImpar}
T7 & Facilidad de uso                 & 0.5 & 5 & 2.5 \\
T8 & Portabilidad                     & 2.0 & 5 & 10.0 \\
\rowcolor{grisFilaImpar}
T9 & Facilidad de modificación        & 1.0 & 5 & 5.0 \\
T10 & Concurrencia                    & 1.0 & 0 & 0.0 \\
\rowcolor{grisFilaImpar}
T11 & Características de seguridad    & 1.0 & 0 & 0.0 \\
T12 & Acceso para partes externas     & 1.0 & 0 & 0.0 \\
\rowcolor{grisFilaImpar}
T13 & Facilidad de formación          & 1.0 & 5 & 5.0 \\
\midrule
\multicolumn{4}{r}{\textbf{TF (suma ponderada)}} & \textbf{33.0} \\
\midrule
\multicolumn{5}{r}{$\text{TCF} = 0{,}06 + 0{,}01 \times 33{,}0 = \mathbf{0{,}39}$}\\
\bottomrule
\end{tabular}
\end{table}

\begin{cajaError}
\textbf{Nota importante sobre TCF:} El ATM es de una sola PC, sin distribución (T1=0),
sin requisitos de rendimiento estrictos (T2=0), sin seguridad (T3=0), sin concurrencia
(T10=0). Sin embargo, la portabilidad (Java) y la facilidad de modificación son
fundamentales. Un error frecuente es puntuar T1 o T10 con valores altos en un sistema
que explícitamente es monousuario y no distribuido.
\end{cajaError}

\paragraph{Paso 7: Factor de Entorno (ECF)} \mbox{} \par

\begin{cajaFormula}
\[ \text{TCF} = 0{,}06 + 0{,}01 \times \text{TF} = 0{,}06 + 0{,}01 \times 33 = \mathbf{0{,}39} \]
\end{cajaFormula}





\begin{table}[H]
\centering
\small
\caption{Factores de Entorno -- ATM}
\label{tab:ecf_atm}
\begin{tabular}{c p{5.5cm} c c c}
\toprule
\cabTbl{Factor} & \cabTbl{Descripción} & \cabTbl{Peso ($f_i$)} &
\cabTbl{Evaluación ($v_i$)} & \cabTbl{$f_i\times v_i$} \\
\midrule
\rowcolor{grisFilaImpar}
E1 & Familiaridad con el proceso (RUP) & 1.5 & 3 & 4.5 \\
E2 & Experiencia en el dominio (ATM)   & 0.5 & 1 & 0.5 \\
\rowcolor{grisFilaImpar}
E3 & Experiencia en OO                 & 1.0 & 2 & 2.0 \\
E4 & Capacidad del analista líder      & 0.5 & 3 & 1.5 \\
\rowcolor{grisFilaImpar}
E5 & Motivación                        & 1.0 & 4 & 4.0 \\
E6 & Estabilidad de requisitos         & 2.0 & 4 & 8.0 \\
\rowcolor{grisFilaImpar}
E7 & Trabajo a tiempo parcial          & -1.0 & 3 & -3.0 \\
E8 & Dificultad del lenguaje (Java nuevo) & -1.0 & 5 & -5.0 \\
\midrule
\multicolumn{4}{r}{\textbf{EF (suma ponderada)}} & \textbf{12.5} \\
\midrule
\multicolumn{5}{r}{$\text{ECF} = 1{,}4 - 0{,}03 \times 12{,}5 = \mathbf{1{,}025}$}\\
\bottomrule
\end{tabular}
\end{table}

\begin{cajaFormula}
\[ \text{ECF} = 1{,}4 - 0{,}03 \times \text{EF} = 1{,}4 - 0{,}03 \times 12{,}5 = \mathbf{1{,}025} \]
\end{cajaFormula}

\paragraph{Paso 8: PCU Ajustados (AUCP)}  \mbox{} \par

\begin{cajaFormula}
\[ \text{AUCP} = \text{UUCP} \times \text{TCF} \times \text{ECF}
               = 35 \times 0{,}39 \times 1{,}025 = \mathbf{13{,}97} \]
\end{cajaFormula}

\paragraph{Pasos 9 y 10: Estimación del Esfuerzo y Duración}  \mbox{} \par

\begin{table}[H]
\centering
\small
\caption{Estimación del Esfuerzo de Codificación -- ATM}
\label{tab:esfuerzo_atm}
\begin{tabular}{l c c c c}
\toprule
\cabTbl{Modelo} & \cabTbl{Productividad} & \cabTbl{Esfuerzo Cod.}
& \cabTbl{Esfuerzo Total*} & \cabTbl{Duración (1 persona)} \\
\midrule
\rowcolor{grisFilaImpar}
Karner         & 20 h/PCU & $13{,}97 \times 20 = 279{,}4$\,hh
               & $279{,}4/0{,}4 = 698{,}5$\,hh $= 3{,}97$\,mh & 3,97 meses \\
Schneider\,\&\,Winters & 28 h/PCU & $13{,}97 \times 28 = 391{,}2$\,hh
               & $391{,}2/0{,}4 = 978{,}0$\,hh $= 5{,}56$\,mh & 5,56 meses \\
\rowcolor{grisFilaImpar}
Propio         & 4 h/PCU  & $13{,}97 \times 4 = 55{,}9$\,hh
               & $55{,}9/0{,}4 = 139{,}7$\,hh $= 0{,}79$\,mh & 0,79 meses \\
\bottomrule
\end{tabular}
\end{table}

\footnotesize{* La codificación representa el 40\% del esfuerzo total
($\text{Esfuerzo total} = \text{Esfuerzo Cod.}/0{,}4$). Tasa: 176\,hh = 1\,mes-hombre.}

\normalsize
\begin{cajaError}
\textbf{Errores frecuentes -- Problema 2 (ATM):}
\begin{itemize}[leftmargin=0.8cm, itemsep=0.2em]
  \item \textbf{A2-Error 1:} Olvidar que el esfuerzo calculado con la productividad
        es \emph{sólo el de codificación}, que representa el \textbf{40\%} del esfuerzo
        total. El total se obtiene dividiendo entre $0{,}4$.
  \item \textbf{A2-Error 2:} Confundir unidades: la \emph{duración} es en
        \textbf{meses}, \emph{no} en meses-hombre. La fórmula es
        $\text{Duración} = \text{Esfuerzo}/\text{N° personas}$, y las
        unidades resultantes son $[\text{mes-persona}/\text{persona}] = [\text{mes}]$.
  \item \textbf{A2-Error 3:} No considerar el actor Administrador si apareciera;
        es un error frecuente omitir actores del diagrama al hacer la clasificación.
\end{itemize}
\end{cajaError}
\fi
% ─────────────────────────────────────────────────────────────────────────────

% ═══════════════════════════════════════════════════════════════════════════════
\subsection*{5.3 \quad Problema 3: Sistema TPV Simplificado}
% ═══════════════════════════════════════════════════════════════════════════════

Obtener los PCU ajustados para el sistema TPV descrito a continuación.
Estimar el esfuerzo y la duración del proyecto.

% ─── 5.3.1 Definición ────────────────────────────────────────────────────────
\subsubsection*{5.3.1 \quad Definición del Problema}

El Sr.~Vicente Pastor, dueño y administrador de un negocio minorista,
contrata una pequeña empresa de software para desarrollar un
\textbf{Terminal de Punto de Venta (TPV)}.

El TPV es un sistema computarizado para registrar ventas y gestionar pagos.
Incluye hardware (computadora y lector de tarjetas) y software. Su objetivo
es proporcionar pago rápido, análisis exacto de ventas y control automático
de existencias.

% ─── 5.3.2 Requisitos ────────────────────────────────────────────────────────
\subsubsection*{5.3.2 \quad Requisitos}

\noindent\textbf{Funciones básicas:}
\begin{itemize}[leftmargin=1.5cm, label=--]
    \item Registrar venta de productos con código universal de producto (CUP).
    \item Calcular el total de la venta y reducir stock.
    \item Mostrar descripción y precio de cada producto.
    \item Generar informes de ventas totales.
\end{itemize}

\noindent\textbf{Funciones de pago:}
\begin{itemize}[leftmargin=1.5cm, label=--]
    \item Gestionar pagos con tarjeta de crédito, autorizados por sistema externo.
    \item Registrar pagos en sistema Cuentas por Cobrar (externo).
\end{itemize}

\noindent\textbf{Restricciones y prestaciones:}
\begin{itemize}[leftmargin=1.5cm, label=--]
    \item Multiusuario con mecanismos de comunicación entre sistemas.
    \item Tiempo de respuesta $<5$ segundos.
    \item Manejable con teclado y ratón. Tolerante a fallos.
\end{itemize}

% ─── 5.3.3 Características del Grupo ─────────────────────────────────────────
\subsubsection*{5.3.3 \quad Características del Grupo de Desarrollo}

\begin{cajaEnunciado}[title=Perfil del equipo -- TPV]
\begin{itemize}[leftmargin=1cm, noitemsep]
    \item Familiarizados con RUP; han desarrollado sistemas similares antes.
    \item Paradigma OO (experiencia alta).
    \item Líder del grupo con \emph{escasa experiencia}; equipo entusiasta.
    \item Requisitos relativamente estables (se esperan algunos cambios).
    \item Todo el grupo trabaja a \textbf{tiempo parcial} (\emph{part-time}).
    \item Muy experimentados en el lenguaje de programación a usar.
\end{itemize}
\end{cajaEnunciado}

% ─── 5.3.4 Diagrama de CU ────────────────────────────────────────────────────
\subsubsection*{5.3.4 \quad Diagrama de Casos de Uso del TPV}

\begin{figure}[H]
    \centering
    \includegraphics[width=0.90\textwidth]{images/diagrama_cu_tpv.png}
    \caption{Diagrama de Casos de Uso del Sistema TPV Simplificado}
    \label{fig:cu_tpv}
\end{figure}

% ─── 5.3.5 Especificación de CU ──────────────────────────────────────────────
\subsubsection*{5.3.5 \quad Especificación de Casos de Uso}

\noindent\textbf{CU\#01 -- Comprar Producto}\\[2pt]
\begin{tabular}{@{}p{3cm}p{11cm}@{}}
\textit{Actor principal:} & Cajero \\
\textit{Precondiciones:} & El Sistema está en espera. \\
\textit{Disparador:} & Se presenta un Cliente. \\
\end{tabular}
\begin{enumerate}[label=\arabic*., leftmargin=1.5cm, noitemsep]
    \item El Cajero introduce código y cantidad de cada producto.
    \item El Sistema determina el precio, incorpora descripción y calcula total.
    \item El Cliente paga con tarjeta de crédito. Sigue en CU02.
    \item El Sistema registra la venta en ``Ventas'' (ext.), descuenta stock y genera recibo.
\end{enumerate}
\textit{Extensiones:}
\begin{itemize}[leftmargin=1.5cm, noitemsep, label=--]
    \item \textbf{1a.} Código incorrecto $\to$ aviso al Cajero; continúa en paso 1.
    \item \textbf{3a.} No puede pagar con tarjeta $\to$ Cajero cancela; continúa en paso 1.
\end{itemize}

\medskip\noindent\textbf{CU\#02 -- Pagar con Tarjeta de Crédito}\\[2pt]
\begin{tabular}{@{}p{3cm}p{11cm}@{}}
\textit{Actor principal:} & Cajero \\
\textit{Precondiciones:} & El Sistema ha calculado el total; espera la tarjeta. \\
\textit{Disparador:} & El Cliente entrega su tarjeta al Cajero. \\
\end{tabular}
\begin{enumerate}[label=\arabic*., leftmargin=1.5cm, noitemsep]
    \item Cajero introduce datos de la tarjeta (tipo, número, cuotas).
    \item Sistema solicita autorización al Sistema de Autorización Externo; éste autoriza.
    \item Sistema registra el pago en Cuentas por Cobrar (ext.) y muestra autorización. Continúa en CU01 paso 4.
\end{enumerate}
\textit{Extensiones:}
\begin{itemize}[leftmargin=1.5cm, noitemsep, label=--]
    \item \textbf{2a.} Autorización rechazada $\to$ mensaje; Cajero cancela; continúa en CU01 paso 1.
\end{itemize}

\medskip\noindent\textbf{CU\#03 -- Generar Informe de Ventas Totales}\\[2pt]
\begin{tabular}{@{}p{3cm}p{11cm}@{}}
\textit{Actor principal:} & Administrador \\
\textit{Precondiciones:} & El Sistema está en espera. \\
\textit{Disparador:} & Se presenta el Administrador. \\
\end{tabular}
\begin{enumerate}[label=\arabic*., leftmargin=1.5cm, noitemsep]
    \item Administrador solicita informe, introduce usuario/contraseña; Sistema valida y da acceso.
    \item Administrador selecciona rango de fechas.
    \item Sistema consulta a ``Ventas'' (ext.) la lista de montos por fecha del rango.
    \item Sistema presenta la información visual e impresa. Continúa en CU01 paso 1.
\end{enumerate}
\textit{Extensiones:}
\begin{itemize}[leftmargin=1.5cm, noitemsep, label=--]
    \item \textbf{1a.} Credenciales inválidas $\to$ mensaje; continúa en paso 1.
    \item \textbf{2a.} Rango de fechas inválido $\to$ mensaje; continúa en paso 2.
\end{itemize}

% ─────────────────────────────────────────────────────────────────────────────
\ifresolucion
% ─────────────────────────────────────────────────────────────────────────────
\subsubsection*{Resolución -- Problema 3: Sistema TPV}

\paragraph{Pasos 1 y 2: Actores y UAW}

\begin{table}[H]
\centering
\small
\caption{Clasificación de Actores -- TPV}
\label{tab:actores_tpv}
\begin{tabular}{p{3.8cm} p{4.5cm} c c c}
\toprule
\cabTbl{Actor} & \cabTbl{Tipo de Interfaz} & \cabTbl{Categoría} &
\cabTbl{Peso} & \cabTbl{Cantidad} \\
\midrule
\rowcolor{grisFilaImpar}
Cajero & Interfaz gráfica (GUI) & Complejo & 3 & 1 \\
Administrador & Interfaz gráfica (GUI) & Complejo & 3 & 1 \\
\rowcolor{grisFilaImpar}
Sist. Ventas & Sistema externo (API) & Simple & 1 & 1 \\
Sist. Autorización & Sistema externo (API) & Simple & 1 & 1 \\
\rowcolor{grisFilaImpar}
Cuentas por Cobrar & Sistema externo (API) & Simple & 1 & 1 \\
\midrule
\multicolumn{4}{r}{\textbf{UAW}} & \textbf{9} \\
\bottomrule
\end{tabular}
\end{table}

\begin{cajaError}
\textbf{Error frecuente A3:} Omitir el actor \textbf{Administrador} (peso = 3) de la
tabla de actores. Es un error documentado en las correcciones: el Administrador
interactúa via GUI en CU03, por lo que es un actor Complejo. Olvidarlo reduce el
UAW en 3 puntos y subestima el proyecto.
\end{cajaError}

\begin{cajaFormula}
\[ \text{UAW} = 1\times 3 + 1\times 3 + 1\times 1 + 1\times 1 + 1\times 1 = \mathbf{9} \]
\end{cajaFormula}

\paragraph{Pasos 3 y 4: Casos de Uso y UUCW}

\begin{table}[H]
\centering
\small
\caption{Clasificación de Casos de Uso -- TPV}
\label{tab:cu_tpv}
\begin{tabular}{p{0.7cm} p{3.8cm} p{4.5cm} c c c}
\toprule
\cabTbl{CU\#} & \cabTbl{Nombre} & \cabTbl{Justificación} &
\cabTbl{Tx} & \cabTbl{Categoría} & \cabTbl{Peso} \\
\midrule
\rowcolor{grisFilaImpar}
01 & Comprar Producto & 3 pasos en EPE (antes de include CU02) & 3 & Simple & 5 \\
02 & Pagar con Tarjeta & 3 pasos principales (entrar datos, autorizar, registrar) & 3 & Simple & 5 \\
\rowcolor{grisFilaImpar}
03 & Generar Informe   & 3 pasos en EPE (autenticar, seleccionar fechas, presentar) & 3 & Simple & 5 \\
\midrule
\multicolumn{5}{r}{\textbf{UUCW}} & \textbf{15} \\
\bottomrule
\end{tabular}
\end{table}

\textit{Nota:} Si CU02 se considera con mayor detalle (datos de tarjeta + validación +
registro en Cuentas Cobrar = 4 transacciones), se clasifica como \textbf{Medio} (10 pts),
lo que haría UUCW = 20. Ambas interpretaciones son válidas; se documenta la preferencia
del equipo.

\begin{cajaFormula}
\[ \text{UUCP} = \text{UAW} + \text{UUCW} = 9 + 15 = \mathbf{24}
   \quad\text{(ó 29 si CU02 es Medio)} \]
\end{cajaFormula}

\paragraph{Paso 6: Factor Técnico (TCF) -- TPV}

A diferencia del ATM, el TPV tiene distribución, multiusuario, rendimiento estricto
y múltiples sistemas externos.

\begin{table}[H]
\centering
\small
\caption{Factores de Complejidad Técnica -- TPV}
\label{tab:tcf_tpv}
\begin{tabular}{c p{5.5cm} c c c}
\toprule
\cabTbl{Factor} & \cabTbl{Descripción} & \cabTbl{Peso} &
\cabTbl{Eval.} & \cabTbl{Pond.} \\
\midrule
\rowcolor{grisFilaImpar}
T1  & Sistema distribuido (múlt. ext.)  & 2.0 & 6  & 12.0 \\
T2  & Rendimiento (resp. $<5$\,s)        & 1.0 & 10 & 10.0 \\
\rowcolor{grisFilaImpar}
T3  & Eficiencia para el usuario        & 1.0 & 5  & 5.0 \\
T4  & Procesamiento interno complejo    & 1.0 & 3  & 3.0 \\
\rowcolor{grisFilaImpar}
T5  & Código reutilizable               & 1.0 & 3  & 3.0 \\
T6  & Facilidad de instalación          & 0.5 & 3  & 1.5 \\
\rowcolor{grisFilaImpar}
T7  & Facilidad de uso                  & 0.5 & 5  & 2.5 \\
T8  & Portabilidad                      & 2.0 & 0  & 0.0 \\
\rowcolor{grisFilaImpar}
T9  & Facilidad de modificación         & 1.0 & 4  & 4.0 \\
T10 & Concurrencia (multiusuario)       & 1.0 & 5  & 5.0 \\
\rowcolor{grisFilaImpar}
T11 & Seguridad (tarjetas)              & 1.0 & 5  & 5.0 \\
T12 & Acceso a código de terceros       & 1.0 & 5  & 5.0 \\
\rowcolor{grisFilaImpar}
T13 & Formación del usuario             & 1.0 & 3  & 3.0 \\
\midrule
\multicolumn{4}{r}{\textbf{TF}} & \textbf{59.0} \\
\midrule
\multicolumn{5}{r}{$\text{TCF} = 0{,}06 + 0{,}01 \times 59{,}0 = \mathbf{0{,}65}$}\\
\bottomrule
\end{tabular}
\end{table}

\begin{cajaFormula}
\[ \text{TCF} = 0{,}06 + 0{,}01 \times 59 = \mathbf{0{,}65} \]
\end{cajaFormula}

\paragraph{Paso 7: Factor de Entorno (ECF) -- TPV}

\begin{table}[H]
\centering
\small
\caption{Factores de Entorno -- TPV}
\label{tab:ecf_tpv}
\begin{tabular}{c p{5.5cm} c c c}
\toprule
\cabTbl{Factor} & \cabTbl{Descripción} & \cabTbl{Peso} &
\cabTbl{Eval.} & \cabTbl{Pond.} \\
\midrule
\rowcolor{grisFilaImpar}
E1 & Familiaridad con el proceso (RUP)   & 1.5  & 4 &  6.0 \\
E2 & Experiencia en el dominio (TPV)     & 0.5  & 4 &  2.0 \\
\rowcolor{grisFilaImpar}
E3 & Experiencia en OO (alta)            & 1.0  & 5 &  5.0 \\
E4 & Capacidad líder (escasa experiencia)& 0.5  & 1 &  0.5 \\
\rowcolor{grisFilaImpar}
E5 & Motivación (equipo entusiasta)      & 1.0  & 4 &  4.0 \\
E6 & Estabilidad de requisitos           & 2.0  & 3 &  6.0 \\
\rowcolor{grisFilaImpar}
E7 & Trabajo a tiempo parcial            & -1.0 & 3 & -3.0 \\
E8 & Dificultad del lenguaje (experim.)  & -1.0 & 0 &  0.0 \\
\midrule
\multicolumn{4}{r}{\textbf{EF}} & \textbf{20.5} \\
\midrule
\multicolumn{5}{r}{$\text{ECF} = 1{,}4 - 0{,}03 \times 20{,}5 = \mathbf{0{,}785}$}\\
\bottomrule
\end{tabular}
\end{table}

\begin{cajaFormula}
\[ \text{ECF} = 1{,}4 - 0{,}03 \times 20{,}5 = \mathbf{0{,}785} \]
\end{cajaFormula}

\paragraph{Paso 8: PCU Ajustados (AUCP) -- TPV}

\begin{cajaFormula}
\[ \text{AUCP} = 24 \times 0{,}65 \times 0{,}785 = \mathbf{12{,}25}
   \quad\text{(ó } 29\times 0{,}65\times 0{,}785 = 14{,}81\text{ con UUCW=20)} \]
\end{cajaFormula}

\paragraph{Pasos 9 y 10: Estimación del Esfuerzo y Duración -- TPV}

Se trabaja con \textbf{2 personas} a tiempo parcial (grupo pequeño).

\begin{table}[H]
\centering
\small
\caption{Estimación del Esfuerzo y Duración -- TPV (AUCP = 12{,}25)}
\label{tab:esfuerzo_tpv}
\begin{tabular}{l c c c c c}
\toprule
\cabTbl{Modelo} & \cabTbl{h/PCU} & \cabTbl{Esfuerzo Cod.}
& \cabTbl{Esfuerzo Total} & \cabTbl{N° Pers.} & \cabTbl{Duración} \\
\midrule
\rowcolor{grisFilaImpar}
Karner  & 20 & $245{,}0$\,hh & $612{,}5$\,hh = 3,48\,mh & 2 & 1,74 meses \\
Sch\&Win & 28 & $343{,}0$\,hh & $857{,}5$\,hh = 4,87\,mh & 2 & 2,43 meses \\
\rowcolor{grisFilaImpar}
Propio  & 5,6 & $68{,}6$\,hh & $171{,}5$\,hh = 0,97\,mh & 2 & 0,49 meses \\
\bottomrule
\end{tabular}
\end{table}

\begin{cajaError}
\textbf{Errores frecuentes -- Problema 3 (TPV):}
\begin{itemize}[leftmargin=0.8cm, itemsep=0.2em]
  \item \textbf{A3-Error 1 (documentado):} No contabilizar el
        \textbf{actor Administrador} (peso=3) al construir la tabla de actores.
        Verificar siempre que todos los actores que aparecen en los CU estén incluidos.
  \item \textbf{A3-Error 2:} Asignar portabilidad (T8) alta cuando el problema
        dice que el sistema será instalado en un punto de venta fijo $\Rightarrow$ T8=0.
  \item \textbf{A3-Error 3:} No aplicar la corrección del $40\%$ del esfuerzo de
        codificación para obtener el esfuerzo total del proyecto.
  \item \textbf{A3-Error 4:} Dividir el esfuerzo total en meses-hombre entre el
        número de personas para obtener la duración en meses.
        La duración en meses $\neq$ los meses-hombre.
\end{itemize}
\end{cajaError}
\fi
% ─────────────────────────────────────────────────────────────────────────────

% ═══════════════════════════════════════════════════════════════════════════════
\ifresolucion
\subsection*{Resumen Comparativo de Estimaciones}
% ═══════════════════════════════════════════════════════════════════════════════

\begin{table}[H]
\centering
\small
\caption{Cuadro Comparativo -- ATM vs. TPV}
\label{tab:comparativo}
\begin{tabular}{l c c}
\toprule
\cabTbl{Concepto} & \cabTbl{ATM} & \cabTbl{TPV} \\
\midrule
\rowcolor{grisFilaImpar}
UAW (actores ponderados)   & 5  & 9 \\
UUCW (CU ponderados)       & 30 & 15 (ó 20) \\
\rowcolor{grisFilaImpar}
UUCP (sin ajuste)          & 35 & 24 (ó 29) \\
TCF                        & 0,39 & 0,65 \\
\rowcolor{grisFilaImpar}
ECF                        & 1,025 & 0,785 \\
\textbf{AUCP (ajustados)}  & \textbf{13,97} & \textbf{12,25} \\
\rowcolor{grisFilaImpar}
Esfuerzo cod. (Karner 20h) & 279,4\,hh & 245,0\,hh \\
Esfuerzo total (Karner)    & 698,5\,hh & 612,5\,hh \\
\rowcolor{grisFilaImpar}
Duración (Karner, 1 pers.) & 3,97 meses & 1,74 meses (2 pers.)\\
\bottomrule
\end{tabular}
\end{table}

\noindent\textbf{Interpretación:} A pesar de tener menos casos de uso, el TPV resulta
más complejo técnicamente (TCF mayor por distribución, concurrencia y seguridad), pero
el equipo más experimentado (ECF menor) y el mayor tamaño del grupo compensa parcialmente.
\fi

\end{document}
