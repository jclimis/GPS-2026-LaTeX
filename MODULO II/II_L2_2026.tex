\documentclass[11pt, a4paper]{article}
\usepackage{formato_catedra}

% ==========================================================
% CONFIGURACIÓN DE MODO: ¿Cuestionario o Resolución?
% ==========================================================
\newif\ifresolucion
\resoluciontrue % <--- Cambia a \resolucionfalse para ocultar respuestas
% ==========================================================

\Lectura{Lectura II\_L2}

\ifresolucion
    \TituloDocumento{PLANIFICACIÓN Y ESTIMACIÓN ÁGIL \\ {\small (Resolución)}}
\else
    \TituloDocumento{PLANIFICACIÓN Y ESTIMACIÓN ÁGIL \\ {\small (Cuestionario)}}
\fi

\begin{document}

\HacerTitulo

\section*{1 \quad CONTENIDO}
Puntos de Historia, Tiempo Ideal, Velocidad del equipo y técnicas de estimación (Planning Poker).

\section*{2 \quad OBJETIVOS}
\begin{itemize}[leftmargin=1.5cm, label=--]
    \item \textbf{Diferenciar} el tamaño relativo (puntos) del tiempo absoluto (horas).
    \item \textbf{Comprender} la métrica de velocidad como capacidad predictiva.
    \item \textbf{Aplicar} técnicas de consenso para reducir la incertidumbre en estimaciones.
\end{itemize}

\section*{3 \quad METODOLOGÍA}
Lectura técnica y resolución de cuestionario basado en Mike Cohn.

\section*{4 \quad BIBLIOGRAFÍA}
\nocite{cohn2005agile}
\printbibliography[heading=none]

\section*{5 \quad ACTIVIDADES}

\subsection*{5.1 \quad Lectura sobre Mike Cohn: Estimación Ágil}
\begin{enumerate}[label=5.1.\arabic*, leftmargin=1.5cm]

    \item \textbf{¿Qué son los puntos de historia y qué características tienen?} 
    \ifresolucion \\ 
    \textbf{[Cap. 4]} Son una unidad de medida relativa para expresar el tamaño de una historia. Son **relativos** (comparativos), **aditivos** y representan una combinación de **esfuerzo, complejidad y riesgo**. \fi

    \item \textbf{¿Cómo se pueden definir los puntos de historias?} 
    \ifresolucion \\ 
    \textbf{[Cap. 4]} Mediante la comparación con una "historia base" conocida. A las demás se les asigna un valor (usualmente de la serie de Fibonacci) según sean más o menos complejas que la base. \fi

    \item \textbf{¿Qué es la velocidad del equipo?} 
    \ifresolucion \\ 
    \textbf{[Cap. 4]} Es el promedio de puntos de historia que un equipo completa funcionalmente en un Sprint. Sirve como métrica predictiva para la planificación de versiones futuras. \fi

    \item \textbf{¿Qué es tiempo ideal y qué diferencia hay con el tiempo transcurrido?} 
    \ifresolucion \\ 
    \textbf{[Cap. 5]} El **tiempo ideal** es el esfuerzo neto sin interrupciones. El **tiempo transcurrido** es el tiempo real del reloj que incluye reuniones, correos y otras distracciones. \fi

    \item \textbf{¿Qué relación existe entre la precisión de la estimación y el esfuerzo?} 
    \ifresolucion \\ 
    \textbf{[Cap. 6]} Invertir demasiado esfuerzo en estimar no garantiza precisión absoluta. La estimación ágil busca ser "suficientemente buena" para tomar decisiones, evitando el desperdicio de tiempo en análisis excesivo. 
    
     \fi

    \item \textbf{¿Qué técnicas de estimación se utilizan en Ágil?} 
    \ifresolucion 
    \begin{itemize}
        \item \textbf{Planning Poker:} Basada en consenso y cartas de Fibonacci.
        \item \textbf{T-Shirt Sizing:} Categorización por tallas (XS a XL).
        \item \textbf{Afining Mapping:} Agrupación visual por similitud de tamaño.
    \end{itemize} \fi

    \item \textbf{¿Qué ventaja tiene el Planning Poker?} 
    \ifresolucion \\ 
    \textbf{[Cap. 6]} Evita el "efecto ancla", fomenta la discusión técnica y revela riesgos u omisiones al obligar a los extremos (estimaciones más altas y bajas) a justificar su postura. \fi

    \item \textbf{¿La estimación por Planning Poker debe ser precisa?} 
    \ifresolucion \\ 
    No. Debe ser **útil**. El objetivo es lograr un entendimiento compartido del trabajo y un compromiso realista del equipo, no una cifra exacta de tiempo. \fi

    \item \textbf{¿Cuándo y por qué es recomendable jugar a Planning Poker?} 
    \ifresolucion \\ 
    Durante el refinamiento del backlog o la planificación del Sprint. Es recomendable porque involucra a quienes harán el trabajo, democratiza la decisión y mejora la calidad de la solución técnica mediante el debate.
    
     \fi

\end{enumerate}

\end{document}