\documentclass[11pt, a4paper]{article}
\usepackage{formato_catedra}

% ==========================================================
% CONFIGURACIÓN DE MODO: ¿Cuestionario o Resolución?
% ==========================================================
\newif\ifresolucion
\resoluciontrue % <--- Cambia a \resolucionfalse para ocultar respuestas
% ==========================================================

\Lectura{Lectura I\_L1}

% Lógica de Título con subtítulo de estado
\ifresolucion
    \TituloDocumento{PRODUCTOS Y PROCESOS \\ {\small (Resolución)}}
\else
    \TituloDocumento{PRODUCTOS Y PROCESOS \\ {\small (Cuestionario)}}
\fi

\begin{document}

\HacerTitulo

\section*{1 \quad CONTENIDO}
Naturaleza del software, dominios de aplicación, sistemas heredados y nuevos paradigmas tecnológicos.

\section*{2 \quad OBJETIVOS}
\begin{itemize}[leftmargin=1.5cm, label=--]
    \item \textbf{Diferenciar} el comportamiento del software frente al hardware.
    \item \textbf{Identificar} los componentes de la definición formal de software.
    \item \textbf{Analizar} los desafíos de los sistemas heredados y entornos modernos.
\end{itemize}

\section*{3 \quad METODOLOGÍA}
Lectura técnica y resolución de cuestionario basado en Pressman (9na Ed.).

\section*{4 \quad BIBLIOGRAFÍA}
\nocite{pressman2021_cap1}

\bibliographystyle{plain}
\bibliography{referencias}


\section*{5 \quad ACTIVIDADES}

\subsection*{5.1 \quad Preguntas sobre Pressman (9na Edición)}
\begin{enumerate}[label=5.1.\arabic*, leftmargin=1.5cm]
    \item \textbf{¿Qué distribuye el software?} 
    \ifresolucion \\ 
    \textbf{[Cap. 1: El software y la ingeniería de software]} El software actúa como un transformador que distribuye el producto más importante de la era moderna: la información. Produce, administra, adquiere, modifica o transmite datos en redes mundiales. \fi

    \item \textbf{¿Cómo es la tasa de fallas del hardware y qué diferencia hay con la del software?} 
    \ifresolucion \\ 
    \textbf{[Cap. 1]} El hardware sigue la "curva de bañera" (desgaste físico). El software no se desgasta físicamente; su tasa de fallas se eleva debido al deterioro lógico causado por cambios que introducen efectos secundarios. \fi

    \item \textbf{¿Cuál es la definición de software?} 
    \ifresolucion \\ 
    \textbf{[Cap. 1]} Se define como un conjunto de: 1) instrucciones (programas); 2) estructuras de datos para la manipulación de información; y 3) información descriptiva (documentación). \fi

    \item \textbf{¿Qué ha llevado al desarrollo de software a adoptar prácticas de ingeniería de software?} 
    \ifresolucion \\ 
    \textbf{[Cap. 1]} La creciente complejidad de los sistemas, la necesidad de gestionar costos y plazos, y la criticidad del software en funciones vitales para la sociedad y los negocios. \fi

    \item \textbf{Ejemplos de aplicaciones de software según los dominios mencionados:} 
    \ifresolucion 
    \begin{itemize}
        \item \textbf{Sistemas:} Sistemas operativos.
        \item \textbf{Aplicación:} Procesadores de texto o navegadores.
        \item \textbf{Ingeniería/ciencias:} Software de astronomía o CAD.
        \item \textbf{Embebido:} Control de microondas o frenos ABS.
        \item \textbf{IA:} Redes neuronales o sistemas expertos.
    \end{itemize} \fi

    \item \textbf{¿Por qué es difícil cambiar un sistema heredado?} 
    \ifresolucion \\ 
    \textbf{[Cap. 1]} Debido a su diseño deficiente, falta de documentación coherente, código altamente acoplado y el riesgo de generar fallas inesperadas al realizar modificaciones. \fi

    \item \textbf{¿Qué características tiene una aplicación móvil?} 
    \ifresolucion \\ 
    \textbf{[Cap. 1]} Interfaces táctiles, conectividad intermitente, recursos de hardware limitados (batería) y arquitecturas optimizadas para multitarea. \fi

    \item \textbf{¿Qué es el Cloud Computing?} 
    \ifresolucion \\ 
    \textbf{[Cap. 1]} Es un modelo que proporciona acceso bajo demanda a un ecosistema compartido de recursos de computación (servidores, redes, almacenamiento) a través de Internet. \fi
\end{enumerate}

\end{document}