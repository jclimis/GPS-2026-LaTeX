\documentclass[11pt, a4paper]{article}
\usepackage{formato_catedra}

% ==========================================================
% CONFIGURACIÓN DE MODO: ¿Cuestionario o Resolución?
% ==========================================================
\newif\ifresolucion
\resoluciontrue % <--- Cambia a \resolucionfalse para ocultar respuestas
% ==========================================================

\Lectura{Lectura I\_L4}

\ifresolucion
    \TituloDocumento{METODOLOGÍAS ÁGILES \\ {\small (Resolución)}}
\else
    \TituloDocumento{METODOLOGÍAS ÁGILES \\ {\small (Cuestionario)}}
\fi

\begin{document}

\HacerTitulo

\section*{1 \quad CONTENIDO}
Manifiesto Ágil, marco de trabajo Scrum (roles y eventos) y prácticas de Extreme Programming (XP).

\section*{2 \quad OBJETIVOS}
\begin{itemize}[leftmargin=1.5cm, label=--]
    \item \textbf{Analizar} los valores y principios del desarrollo ágil.
    \item \textbf{Identificar} roles, artefactos y dinámicas en Scrum.
    \item \textbf{Diferenciar} las prácticas técnicas y roles específicos de XP.
\end{itemize}

\section*{3 \quad METODOLOGÍA}
Lectura técnica y resolución de cuestionario basado en Smith, Schwaber y Beck.

\section*{4 \quad BIBLIOGRAFÍA}
\nocite{smith2009, schwaber2004, beck_xp}

\bibliographystyle{plain}
\bibliography{referencias}


\section*{5 \quad ACTIVIDADES}

\subsection*{5.1 \quad Lectura sobre Greg Smith (Agile)}
\begin{enumerate}[label=5.1.\arabic*, leftmargin=1.5cm]
    \item \textbf{¿Cómo debe ser el ciclo de vida actual?} 
    \ifresolucion \\ Debe ser flexible y adaptativo, permitiendo que el equipo aprenda de requisitos emergentes y ajuste el producto mediante iteraciones cortas en lugar de predicciones estáticas. \fi

    \item \textbf{¿Cuáles son “Los Cuatro Valores“ del Manifiesto Ágil?} 
    \ifresolucion \\ 1. Individuos e interacciones sobre procesos y herramientas. 2. Software funcionando sobre documentación exhaustiva. 3. Colaboración con el cliente sobre negociación contractual. 4. Respuesta ante el cambio sobre seguir un plan. \fi

    \item \textbf{Defina en no más de una oración los puntos clave de los principios ágiles.} 
    \ifresolucion 
    \begin{itemize}
        \item Priorizar la comunicación humana para resolver problemas complejos.
        \item El valor real es una aplicación operativa, no manuales teóricos.
        \item Eliminar barreras trabajando codo a codo con el cliente.
        \item Aceptar que el aprendizaje es más valioso que un plan inicial.
    \end{itemize} \fi

    \item \textbf{Realice una comparación enfoque tradicional vs. enfoque ágil.} 
    \ifresolucion \\ El tradicional es predictivo e intenta "congelar" requisitos al inicio; el ágil es adaptativo y usa ciclos de retroalimentación para refinar el conocimiento evolutivo. \fi
\end{enumerate}

\subsection*{5.2 \quad Lectura sobre Ken Schwaber (Scrum)}
\begin{enumerate}[label=5.2.\arabic*, leftmargin=1.5cm]
    \item \textbf{¿Qué es un proyecto de desarrollo de software según el autor?} 
    \ifresolucion \\ Es un proceso empírico complejo (caos controlado) que requiere inspección constante y adaptación de resultados para manejar la incertidumbre. \fi

    \item \textbf{¿Cuál es la tercera dimensión del gráfico de evaluación de complejidades?} 
    \ifresolucion \\ La dimensión de las **Personas**, ya que el éxito depende de la interacción humana frente a tecnologías y requisitos cambiantes. \fi

    \item \textbf{¿Cuál es el esqueleto y cuál es el corazón de SCRUM?} 
    \ifresolucion \\ El esqueleto son los pilares: Transparencia, Inspección y Adaptación. El corazón es el **Sprint**, donde se transforma una idea en un incremento de producto funcional. \fi

    \item \textbf{¿Cuáles son los roles en un Scrum Team?} 
    \ifresolucion 
    \begin{itemize}
        \item \textbf{Product Owner:} Responsable de la visión y prioridades del negocio.
        \item \textbf{Scrum Master:} Facilitador que elimina impedimentos y asegura el proceso.
        \item \textbf{Development Team:} Profesionales auto-organizados que ejecutan el trabajo técnico.
    \end{itemize} \fi
\end{enumerate}

\subsection*{5.3 \quad Lectura sobre Kent Beck (XP)}
\begin{enumerate}[label=5.3.\arabic*, leftmargin=1.5cm]
    \item \textbf{¿Cuáles son las habilidades que debe poseer un programador XP?} 
    \ifresolucion \\ Dominio de diseño simple, programación en parejas, refactorización constante y Test-Driven Development (TDD). \fi

    \item \textbf{¿Cuál es la diferencia entre un cliente XP y uno tradicional?} 
    \ifresolucion \\ El cliente XP es un miembro activo que define historias de usuario en tiempo real, no un agente externo que solo entrega un pliego de condiciones. \fi

    \item \textbf{¿Cuál es la diferencia entre el Preparador y el Controlador (Tracker)?} 
    \ifresolucion \\ El Preparador facilita la infraestructura técnica, mientras que el Controlador mide objetivamente el progreso y la velocidad mediante métricas. \fi

    \item \textbf{¿En qué situaciones el equipo necesita un Consultor?} 
    \ifresolucion \\ Cuando surgen problemas técnicos especializados que exceden la capacidad actual del equipo y requieren intervención externa puntual. \fi
\end{enumerate}

\end{document}